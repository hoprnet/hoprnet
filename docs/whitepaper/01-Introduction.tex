\section{Introduction}

The \MYhref{http://gavwood.com/web3lt.html}{Web3} comprises a range of technologies that enable truly decentralized applications (dApps) featuring increased privacy and resilience for the next wave of web applications. dApps on the Web3 do not rely on central infrastructure with central points of failure but instead support self-custody of wealth and data. Beyond technical innovation, the Web3 ecosystems strives to innovate the business and organizational status quo of today's web2.0. Are rent-seeking for-profit corporations with shareholder meetings and CEOs the only possible way to generate value for users? Are shareholders necessarily the only financial beneficiaries of a project while most work and value might be community-created? The \MYhref{https://www.forbes.com/sites/chancebarnett/2017/09/23/inside-the-meteoric-rise-of-icos/\#76ac96d15670}{ICO wave of 2017} has not just created an alternative funding scheme for high risk early investments via tokens but also led to an increasing number of projects that aspire to the Web3 vision and build its infrastructure. HOPR believes in that privacy-first, decentralized and self-empowering Web3 vision and delivers a privacy foundation for the Web3.

\subsection{Pillars of the web3}
While several components of the web3 are still heavily in the making, some projects are already today giving a glimpse of that future web architecture. Early examples working towards dApps are decentralized organizations (DAOs, e.g. by \MYhref{https://aragon.org}{Aragon}), social media platforms such as \MYhref{https://akasha.world}{Akasha} or financial products like \MYhref{https://makerdao.com/en/dai}{Maker's DAI} or non-custodial trading venues like \MYhref{https://uniswap.exchange}{Uniswap}. Currently, we see technological pillars emerging that enable developers to build true dApps which do not rely on central infrastructure anymore:
\begin{itemize}
    \item Financial asset management systems enabled by blockchains such as \MYhref{https://ethereum.org/}{Ethe\-reum} or \MYhref{https://z.cash/}{ZCash}
    \item Data storage solutions like \MYhref{https://filecoin.io/}{Filecoin} or \MYhref{https://www.nucypher.com/}{NuCypher}
    \item Computation providers like \MYhref{https://golem.network/}{Golem} or \MYhref{https://enigma.co}{Enigma}
\end{itemize}

\subsection{The web3 stack}
In this new ecosystem, multiple decentralized applications (dApps) interact with one another as well as with these core technologies. All pillars feature projects that focus on privacy within that domain. For example, we see on-chain privacy solutions such as \MYhref{https://www.aztecprotocol.com}{AZTEC} and \MYhref{http://matterlabs.dev}{MatterLabs} on Ethereum or ZCash, private data storage by re-encryption in NuCypher and privacy-preserving computation in Enigma. At the same time, the ecosystem is lacking a go-to solution for network-level privacy enabling communication between separate networks, applications and users. Some dApps make use of \MYhref{https://github.com/ethereum/wiki/wiki/Whisper}{Whisper} which is developed by the Ethereum community but which - similar to other broadcast schemes - suffers from scalability restrictions when used for point-to-point communication and unclear delivery behavior. We build HOPR as a metadata-private communication foundation for the Web3 and the web of today. As such, HOPR solves the privacy fallacy of end-to-end encryption that we see in today's web applications: While the encrypted  message itself might not be accessible to third parties, metadata such as "Who are you talking to?", "How often are you talking to them?", "From where?", "At what time?", "How many and long messages do you exchange?" and more, are leaking and accessible to various third parties including wifi-providers, internet service providers, device manufacturers and other low-level hard- and software vendors.

\setlength{\tabcolsep}{1em} % for the horizontal padding
{\renewcommand{\arraystretch}{2}% for the vertical padding
\begin{center}
    \begin{tabular}{|c|c|c|}
        \hline
        \multicolumn{3}{|c|}{\textbf{dApps}} \\
        \hline
        \makecell{\textbf{Assets}\\Bitcoin, ZCash, tokens} & \makecell{\textbf{Storage}\\IPFS, NuCypher} & \makecell{\textbf{Computation}\\Golem, Enigma}\\
        \hline
        \multicolumn{3}{|c|}{\makecell{\textbf{Messaging}\\\textbf{HOPR}}} \\
        \hline
    \end{tabular}
\end{center}
}

\subsection{Protocol layers of the web3}
HOPR fills the gap between peer-to-peer (P2P) networks and dApps that exchange sensitive information. It adds metadata privacy on top of an existing P2P layer that is used in form of \MYhref{https://libp2p.io/}{libp2p} or \MYhref{https://en.wikipedia.org/wiki/WebRTC}{WebRTC} in decentralized architectures today. It is compatible with underlying network protocols such as \MYhref{https://en.wikipedia.org/wiki/Internet_protocol_suite}{TCP/IP} or \MYhref{https://en.wikipedia.org/wiki/QUIC}{QUIC}. Depending on the application, one layer above HOPR could be an optional storage / sync layer like \MYhref{https://matrix.org/}{Matrix} which then enables e.g. chat application with longer-term message caching.

\setlength{\tabcolsep}{1em} % for the horizontal padding
{\renewcommand{\arraystretch}{2}% for the vertical padding
\begin{center}
    \begin{tabular}{|l|l|l|}
        \hline
        \textbf{Layer} & \textbf{Purpose} & \textbf{Example}\\
        \hline 
        Application & Application logic & \makecell[l]{Chat app,\\M2M comms}\\
        \hline
        Storage / sync & \makecell[l]{Synchronization of data,\\version management,\\medium-term message caching} & Matrix\\
        \hline
        \textbf{Privacy} & \textbf{\makecell[l]{Scalable \& decentralized\\metadata protection,\\incentivization for\\packet relayers,\\short-term caching}} & \textbf{HOPR}\\
        \hline
        P2P & \makecell[l]{Overlay routing,\\NAT traversal} & libp2p, WebRTC\\
        \hline
        Network & \makecell[l]{Underlay routing,\\congestion control} & TCP/IP, QUIC\\
        \hline
    \end{tabular}
\end{center}
}
In the blockchain world, HOPR complements technologies that provide on-chain privacy. Data packets produced by dApps may not just contain valuable data but may also reveal metadata that can be linked to real-world identities. On-chain privacy, for example, is of limited impact if a network observer can link metadata to a social media account in order to attack the person because of the fact \textit{that} they used privacy-preserving financial networks - without knowing \textit{what} exactly they used them for.
