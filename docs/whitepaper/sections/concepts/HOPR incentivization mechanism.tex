\subsection{HOPR incentivization mechanism}
HOPR incentivizes nodes in order to achieve correct transformation and delivery of mixnet packets. 
This is accomplished using a mechanism called “Proof-Of-Relay” with the following second-layer solutions which are both cost effective and privacy preserving.

\subsubsection{Probabilistic payments within payment channels}
In payment channels, two parties A and B lock some funds within a smart contract, make multiple transactions off-chain and only commit the aggregation on-chain. 
This implies that the last HOPR acknowledgement contains all previous incentives plus the incentive for the most recent interaction 
   $$value(ACK_n) =\sum_{i=1}^nfee(packet_i)$$
If B received $ACK_n$ before sending $packet_{n-1}$, it has no incentive to process $packet_{n-1}$ rather than $packet_{n-2}$. 

\subsubsection*{Probabilistic payments}  
The payouts use a concept called “tickets”. In this scheme, there is a known selection rate $s$ (assuming $s=1000$).
A ticket can be a win or a loss with $\frac{1}{1000}$ probability which means nodes are incentivized to continue relaying packets as they don’t know which ticket is a win. 
\\~\\HOPR uses a custom-made second-layer solution. It is inspired by payment channels and probabilistic payments where incentives can be claimed independently: 
$$value ( ACK_i )=value ( ACK_j ) \quad for \quad i,j\in \{1,n\}$$
Hence, there is no added value in pretending packet loss or intentionally changing the order in which packets are processed. 
\subsubsection*{Privacy challenges}
The incentives break the unlinkability guarantees inherited from the SPHINX packet format as they reveal the identity of the packet origin who transfers those incentives in the channel using their signature. 
\\To solve this problem, HOPR forward incentives next to the packet.
\begin{figure}[H]
    \centering
    \includegraphics[width=10cm,height=10cm,keepaspectratio]{../whitepaper/images/token_cashflow.png}
    \caption{Incentive flow}
    \label{fig:Incentive flow}
    \end{figure}

    \hspace{-5mm}This however leaks the relayer’s position within the selected path since the value of the ticket is set according to the current relay fee and the number of intermediate hops, 
more precisely $$amount:=\frac{(hops -1)* relayFee}{winProb}$$
This leakage is considered to have a low severity but further research will be conducted on the subject.







