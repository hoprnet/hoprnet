\section{Peer-to-peer Mechanisms}

HOPR is designed as a decentralized network, hence there is no central coordinator and nodes need to interact with each other directly to organize the operation of the network. This causes various challenges: starting with the absense of a complete overview of the network, so nodes need to deal with incomplete information. Some nodes reside on publicly exposed hosts whilst others are hidden behind one or more network address translation mechanisms (NATs). It is expected that especially those nodes which are run by end users and therefore mostly use the HOPR network to send and receive messages, are subject of churn. For that reason, it becomes necessary to continuously measure the nodes' availability.

\import{}{01-addressing.tex}
\import{}{02-nat-traversal.tex}
\import{}{03-peer-discovery.tex}
\import{}{04-availability.tex}





