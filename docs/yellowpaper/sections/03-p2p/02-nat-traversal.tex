\subsection{NAT Traversal}

It is expected that many, if not most, of the nodes in the HOPR network operate behind one or more network address translation mechanisms (NAT). This can include the address mapping of the ISP if it uses \href{https://en.wikipedia.org/wiki/Carrier-grade_NAT}{carrier-grade NAT}, the router bridging between local network and the internet, as well as containerized environments who add an additional address mapping.

Due to this, it is for most nodes unlikely to establish a direct connection to other nodes without the help of third-parties. By using relay addresses, nodes are able to establish a relayed connection and use it to exchange connection information to determine how to connect directly. To accomplish this, all nodes in the network offer relay services and nodes can register at each other to relay traffic to them.

\begin{tikzpicture}
    \def\nodeHeight{5.4}
    \def\nodeWidth{0.6}
    \def\nodeOffset{4.25}
    \def\oneY{0.3}
    \def\relayPhaseOffset{1.5}

    % Nodes A and B
    \foreach \offset\name\address in{0/A/5.6.7.8,2*\nodeWidth+2*\nodeOffset/B/1.2.3.4} {
            \draw[shift={(\offset,0)}] (0,0) rectangle node[below=80pt] {\small{\shortstack{\name\\\address}}}(\nodeWidth,-\nodeHeight) ;
        }

    % Relay
    \draw (\nodeWidth+\nodeOffset,-\relayPhaseOffset) rectangle (\nodeWidth+\nodeOffset+\nodeWidth,-4.2);

    % TCP phase
    \draw[->,shift={(0,-1*\oneY)}] (\nodeWidth,0) -- (2*\nodeWidth+2*\nodeOffset,-\oneY) node[midway,above,sloped] {\small{1.2.3.4:9091}};
    \draw[->,shift={(0,-3*\oneY)}] (\nodeWidth,0) -- (2*\nodeWidth+2*\nodeOffset,-\oneY) node[midway,above,sloped] {\small{1.2.3.4:50043}};

    % Relay phase
    \begin{scope}[shift={(0,-\relayPhaseOffset)}]
        \draw[->] (\nodeWidth,0) -- (\nodeWidth+\nodeOffset,-0.15) node[midway,above,sloped] {
            \small{relay A}};

        \draw[->,shift={(1*\nodeWidth+\nodeOffset,-0.3)}] (\nodeWidth,0) -- (\nodeWidth+\nodeOffset,-0.15) node[midway,above,sloped] {\small{connect?}};

        \draw[->,shift={(1*\nodeWidth+\nodeOffset,-0.75)}] (\nodeWidth+\nodeOffset,0) -- (\nodeWidth,-0.15) node[midway,above,sloped] {\small{OK}};

        \draw[->] (\nodeOffset+\nodeWidth,-1.05) -- (\nodeWidth,-1.2) node[midway,above,sloped] {\small{Relay OK}};

        \foreach \i in{0,1,2} {
                \begin{scope}[shift={(0,-1.65)}]
                    \ifnum\i=0
                        \draw[->] (\nodeWidth,0) -- (\nodeOffset+\nodeWidth,-0.15) node[midway,above,sloped] {Signalling};
                        \draw[->] (2*\nodeWidth+\nodeOffset,-0.3) -- (2*\nodeOffset+2*\nodeWidth,-0.45) node[midway,above,sloped] {Signalling};
                    \else
                        \draw[->] (\nodeWidth,-\i*0.15) -- (\nodeOffset+\nodeWidth,-0.15-\i*0.15);
                        \draw[->] (2*\nodeWidth+\nodeOffset,-0.3-\i*0.15) -- (2*\nodeOffset+2*\nodeWidth,-0.45-\i*0.15);
                        \draw[->] (2*\nodeWidth+2*\nodeOffset,-0.375-\i*0.15) -- (\nodeOffset+2*\nodeWidth,-0.525-\i*0.15);
                        \draw[->] (*\nodeWidth+*\nodeOffset,-0.525-\i*0.15) -- (\nodeWidth,-0.675-\i*0.15);
                    \fi
                    % \draw[->] (\nodeWidth,-1.65) -- (\nodeOffset+\nodeWidth,-1.8) node[midway,above,sloped] {Signalling};

                \end{scope}
            }


    \end{scope}



\end{tikzpicture}