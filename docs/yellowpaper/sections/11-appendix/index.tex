\begin{comment}


\section{Appendix}
\vspace{1cm}
\begin{tabular}{ c | c | c | c | c}
    Name        & Definition & DataType & Size in bytes & usage \\ \hline
    Recipient   &            & address  &               &       \\
    Amount      &            & uint256  &               &       \\
    TicketIndex &            & uint256  &               &       \\
    Iteration   &            & uint256  &               &       \\
    WinProb     &            & uint256  &               &       \\
    Epoch       &            & uint256  &               &       \\
    Challenge   &            & bytes32  &               &       \\
    ChainId     &            & uint8    & 1             &       \\
    Version     &            & uint8    & 1             &       \\
    Tag         &            & uint8    & 1             &       \\
    SignatureX  &            & bytes32  & 32            &       \\
    SignatureY  &            & bytes32  & 32            &
\end{tabular}
\end{comment}

\section{Key derivation}
\label{sec:keyderivation}

During a protocol execution, a node derives a master secret from the SPHINX header that then used to derive multiple sub-secrets for several purposes by using BLAKE2s as a hash function for HKDF. HKDF is given by two algorithms $\mathsf{extract}$ and $\mathsf{expand}$ where $\mathsf{extract}$ is used to extract the entropy from a given secret $s$, such as an elliptic-curve point, and produces the intermediate keying material (IKM). The IKM then serves as a master secret to feed $\mathsf{expand}$ in order to derive pseudorandom subkeys in the desired length.

\subsection{Extraction}

As a result of the packet creation and its transformation, the sender is able to derive a shared secret $s_i$ given as a compressed elliptic-curve point (33 bytes) with each of the nodes along the path.

$$s_i^{master} \longleftarrow \mathsf{extract}(s_i, 33, pubKey)$$

By adding their own public key $pubKey$ as a salt, each node derives a unique $s_i^{master}$ for each $s_i$.

\subsection{Expansion}

Each subkey $s_i^{sub}$ is used for one purpose, such as keying the \textit{pseudorandomness generator} (PRG).

$$s_i^{sub} \longleftarrow \mathsf{expand}(s_i^{master}, length(purpose), hashKey(purpose))$$

\begin{center}
    \begin{tabular}{|c | c| c | c |}
        \hline
        Usage                           & Purpose         & Length & Hash Key (UTF-8)       \\
        \hline
        \hline
        \multirow{3}{*}{SPHINX packet}  & PRG             & 32     & HASH\_KEY\_PRG         \\
                                        & PRP             & 128    & HASH\_KEY\_PRP         \\
                                        & Packet tag      & 32     & HASH\_KEY\_PACKET\_TAG \\
        \hline
        \multirow{2}{*}{Proof of Relay} & acknowledgement & 32*    & HASH\_KEY\_ACK\_KEY    \\
                                        & ownKey          & 32*    & HASH\_KEY\_OWN\_KEY    \\

        \hline
    \end{tabular}
\end{center}

Since the values for PoR are used as field elements on the curve, there exist a non-zero probability that the application of $\mathsf{expand}$ leads to a value outside of the field. In this case, the hash key is padded by ``\_'' until it leads to a field element.
