\subsubsection{Replay protection}
\label{sec:sphinx:replayprotection}

The creator of a packet picks a path that the packet is supposed to take. As seen in Section \ref{sec:sphinx:keyderivation}, the key derivation is performed such that the packet cannot be processed in any order other than the one chosen by its creator.\footnote{TODO: Make terminology more consistent with regards to ``creator" and ``sender" of a packet.}

This behaviour prevents the adversary from changing the route but also allows no other route. Hence the adversary can be sure that there is no second possible route. Therefore, it can try to replicate the packet and send it via multiple routes to see which connections are used more and thus reveal the route of the packet.

To prevent this attack scenario, each node $n_i$ computes a fingerprint $s_i^{tag}$ of each processed packet and stores it in order to refuse the processing of already seen packets. The value $s_i^{tag}$ is generated from master secret $s_i$ as seen in Section \ref{sec:sphinx:keyderivation} using the key derivation described in Appendix \ref{appendix:keyderivation}.\footnote{TODO: Clarify storage mechanism for these tags and the impact on performance as the network grows.}
