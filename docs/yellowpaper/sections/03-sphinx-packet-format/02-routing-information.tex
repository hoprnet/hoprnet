\subsubsection{Routing information}
Each node on the path needs to know which is the next downstream node. Therefore, the sender $A$ generates routing information $\beta_i$ for $B$, $C$, and $D$, as well as a message $END$ to tell $Z$ that it is the final recipient of the message. Whilst the keys for $B$, $C$, and $D$ are given as a compressed elliptic, the $END$ message occurs as a distinguished prefix.

In general, compression of elliptic curve points given as $(x,y)$ happens by taking the $x$-coordinate and the sign of $y$. If $y$ is above the $x$-axis, \texttt{0x02} is added, otherwise \texttt{0x03} is used. The $END$ message is given by \texttt{0x04} and is therefore easily distinguishable from public keys.

\begin{comment}
The $END$ message is a distinguished prefix byte which is added to the final recipient's compressed public key. For ECDSA public key compression, only the $x$ coordinate is used and is prepended by $02$.

The $y$ coordinate is extracted from $x$ by resolving the secp256k1 elliptic curve equation $Y^2=X^3+7$ \cite{secp}. A square root extraction will yield $Y$ or $-Y$. The compressed point format includes the least significant bit of $Y$ in the first byte (the first byte is $0\times02$ or $0\times03$, depending on that bit).
\end{comment}

The routing information is computed as follows:

\begin{align}
    \beta_{v-1} & =(y_Z\|0_{(2(r-v)+2)\kappa-|y_Z|}\oplus \rho(h_{\rho}(s_{v-1}))_{[ \,0....(2(r-v)+3)\kappa-1\,]})\|\phi_{v-1}                \\
    \text{and}  & \nonumber                                                                                                                    \\
    \beta_i     & =y_{i+1}\|\gamma_{i+1}\|\beta_{{i+1}_{[ \,0....(2r-1)\kappa-1\,] }}\oplus \rho(h_{\rho}(s_{i}))_{[ \,0....(2r+1)\kappa-1\,]}
    \label{eq:2}
\end{align}
$0\le i < v-1$

such that $y_Z$ is the destination's public key in compressed form (since this is only the $x$-coordinate, it is 33 bytes instead of 64) and $|y_Z|$ is its length. $\rho$ is a pseudorandom generator (PRG) and $h_{\rho}$ is the hash function used to key $\rho$.
$v\leq r$ is the length of the path traversed by the packet, where $|y_Z| \leq (2(r - v) + 2)$. $\phi$ is a filler string such that
\begin{align}
    \phi_i & =\{ \phi_{i-1}\|0_{2\kappa}\}\oplus \rho(h_{\rho}(s_{i-1}))_{[ \,(2(r-i)+3)\kappa..(2r+3)\kappa-1\,]}
\end{align}
where $\phi_0=\epsilon$ is an empty string. $\phi_i$ is generated using the shared secret $s_{i-1}$ and used to ensure the header packets remain constant in size as layers of encryption are added or removed. Upon receiving a packet, the processing node extracts the information destined for it from the route information and the per-hop payload. The extraction is performed by deobfuscating and left-shifting the field. Ordinarily, this would make the field shorter at each hop, allowing an attacker to deduce the route length. For this reason, the field is pre-padded before forwarding. Since the padding is part of the HMAC, the origin node will have to pre-generate an identical padding (to that generated at each hop) in order to compute the HMACs correctly for each hop.

$\beta_i$ is computed as the concatenation of $y_Z$ and a sequence of padding which is then encrypted by XORing with the output of a PRG seeded with shared key $s_{v-1}$ of node $v-1$. The result is finally concatenated with $\phi$ to ensure the header packets remain constant in size.

In the original Sphinx paper, $y_Z$ is concatenated with an identifier $I$ and $0$ padding sequence, where $I$ is used for SURBs (single-use reply blocks) such that $I \in \{0, 1\}^\kappa$. We do not use $I$ since HOPR does not currently employ SURBs. We do, however, include $hint$ and $challenge$ values in $\beta$, defined in the \lcnameref{sec:proofofrelay} section. These values are not included in the original Sphinx paper but are needed for the HOPR protocol. Since $A$ has a shared secret with each of the nodes along the path, it is able to derive blindings for each of them. Each node along the path receives an authentication tag $\gamma_i$ in the form of a message authentication code (MAC), which is encoded in the header.

Padding is added at each mix stage in order to keep the length of the message invariant at each hop.

The mix header is constructed as follows:
\begin{align}
    M_i & =(\alpha_i,\beta_i,\gamma_i)
\end{align}

$A$ sends the mix header $M_0$ to $B$. Once $B$ receives the packet, it derives the shared key $s_0$ by computing

$$s_0=(\alpha_0)^b=(g^x)^b=(g^b)^x=y^x_B$$

and removes its blindings. Here $b$ is the private key of node B. This allows $B$ to unblind the routing info that tells $B$ the public key of the next downstream node, $C$. The process happens in the same fashion for all further downstream nodes after $B$.

