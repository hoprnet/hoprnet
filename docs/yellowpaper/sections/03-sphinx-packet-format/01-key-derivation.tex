\subsubsection{Key derivation}
\label{sec:sphinx:keyderivation}

The sender $A$ picks a random $x\in \mathbb{Z}^*_q$ that is used to derive new keys for every packet.

$A$ randomly picks a path consisting of intermediate nodes $B$, $C$, $D$, and the packet's final destination, $Z$.

$A$ performs an offline Diffie-Hellman (DH) key exchange with each of these nodes and derives shared keys with each of them.

$A$ computes a sequence of $r$ tuples (in our case $r$=4)  $$(\alpha_0,s_0,b_0),.................,(\alpha_{r-1},s_{r-1},b_{r-1})$$ as follows:
$$\alpha_0=g^x,s_0=y^x_B,b_0=h_b(a_0,s_0)$$
and
\begin{equation}
    \begin{cases}
        \alpha_i=g^{x\Pi_{j=0}^{j=i-2}b_j} \\
        s_i=y^{x\Pi_{j=0}^{j=i-2}b_j}      \\
        b_i=h_b(a_i,s_i)
    \end{cases}\,.
    \label{eq:1}
\end{equation}
for $1\le i < r-1$, where $y_0, y_1, y_2$, and $y_3$ are the public keys of the nodes $B$, $C$, $D$, and $Z$, respectively, which we assume to be available to $A$ .