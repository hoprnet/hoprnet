\subsection{Opening cover traffic channels}
\label{sec:ct:channelopen}

CT nodes must open and fund payment channels before they can send cover traffic packets via that counterparty. The number of channels a CT node can open is predefined and cannot be exceeded. Whenever payment channels are closed, the CT node replaces them by opening new channels, provided it still has funds. Channel opening is randomly weighted according to the importance of a node. A node's importance, $\Omega(N)$, is defined as follows:

$$\Omega(N) = st(N) * sum(w(C), \forall \; outgoingChannels(N))$$
where
$$st(N) = uT(N) + tB(N)$$
and
$$w(C) = \sqrt{(B(C) / st(Cs) * st(Cd))}$$
\\
where $N$ is the node, $w$ is the channel's weight, $st$ is the number of HOPR tokens staked by the node, and $C$ is a payment channel between two nodes in the HOPR network. For this channel, $Cs$ is the node which opened the payment channel, while $Cd$ is channel counterparty. $uT(N)$ is the balance of unreleased tokens for a node $N$. $B$ is the channel balance and $tB$ is the sum of the outgoing balances of each channel from node $N$.

In accordance with this definition, a node's importance score increases with the channel balances of channels that node shares with other nodes with high total stake. This means that the odds of channel being opened to a particular node is not simply proportional to that node’s stake. Cover traffic packets are allocated in proportion to a node's importance rather than channel stake as a way to mitigate against selfish node operators seeking to maximize their own profits.

The way importance and weight are calculated prevents nodes from opening a few minimally funded channels to nodes with large stake and incentivizes nodes to stake in as many channels as possible,\footnote{TODO: Expand on this concept, demonstrating the optimal number of channels for a node to open at different network sizes and topologies.} including re-allocating their stake to their downstream nodes to receive cover traffic there. This discourages centralization and prevents ``thin traffic",\footnote{TODO: Expand on this concept with more precise definition and threshold.} which limits the size of the anonymity set and increases the risk of traceable packets, even with high per-hop latency.

Since distributing cover traffic comes at a cost to network bandwidth, it is important to try and minimise wasted transmission. Therefore, CT nodes distribute cover traffic based on node importance rather than node uptime. This is because individual nodes are liable to suddenly lose connectivity.\footnote{TODO: Explain how this is accounted for.} Of course, this distribution method will still result in sending cover traffic to offline nodes. If a mix node is found to be offline, the channel to that node is closed and new one to a different node is opened. The offline node will have to wait until the CT node is ready to open a new channel to stand a chance of being reselected.