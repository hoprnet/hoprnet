\subsection{Closing cover traffic channels}
\label{sec:ct:channelclose}

To minimize wasted cover traffic, CT nodes will close any channel deemed likely to result in failed distribution and open new channel in its place. A cover traffic channel is closed and a new one opened if any of the following factors falls below its threshold:

\begin{itemize}
    \item \textbf{Channel balance} The channel balance must remain higher than the minimum stake value, defined as:
          $$B(C) < L * \sigma$$
          where $B(C)$ is the channel balance, $L$ is the path length, and $\sigma$ is the ticket payout amount. This is currently hardcoded as 0.1 HOPR per ticket, but in future versions will be allowed to fluctuate.\footnote{TODO: Expand on the economic and governance factors behind this.}
    \item \textbf{Connection quality} Connection quality, $q$, is defined as the fraction of `pings' the node has responded to within a 5 second cut-off. $q$ must not drop below the defined upper bound.\footnote{TODO: Clarify relationship between $q$ and node heartbeat.}
    \item \textbf{Number of failed packets} The number of failed packets must stay above the failed packet threshold.\footnote{TODO: Clarify and justify threshold, and expand on the implications of this value changing.}
          $$ $$
\end{itemize}

If a node is detected as having fallen below any of these thresholds, the CT node calls \textit{initiateChannelClosure()} and emits two events: \textit{ChannelUpdate} and \textit{ChannelClosureInitiated}.