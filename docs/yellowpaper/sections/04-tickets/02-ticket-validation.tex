\subsection{Ticket Validation}
\label{sec:tickets:validation}

Tickets are used to convince their recipient that they will receive the promised incentive once the challenge is solved. As ticket issuance happens without any on-chain interaction, it is the duty of the recipient to decide whether to accept or refuse a ticket.

Ticket validation runs through two states: receiving the ticket without knowing the response to the given challenge stated as $ticket.challenge$, \lcnameref{sec:tickets:validation:locked}, and once the response is known, \lcnameref{sec:tickets:validation:unlocked}.

\paragraph{Validation of Locked Tickets}
\label{sec:tickets:validation:locked}

Since there is no response to the stated challenge, the node cannot determine whether the ticket is going to be a winner nor claim it on-chain to receive the incentives. Nevertheless, the node can use the embedded information to validate the ticket economically. Therefore, the node first extracts the winning probability as

$$ticket.winProb = \frac{ticket.invWinProb}{2^{256} - 1} $$

which leads to $ value(ticket) = ticket.value \cdot ticket.invWinProb $. If the node considers $value(ticket)$ incorrect, i.e., because it does match the expected amount, or if winning probability is set too high or too low, it should refuse the ticket.

As ticket issuance happens without any on-chain interaction, there is no guarantee that the payment channel has sufficient locked tokens to pay a winning ticket, or indeed that the channel exists at all. Therefore, the recipient must check both of these before considering a ticket valid. In addition, there could be previous tickets, denoted as $stored$, which have not yet been redeemed. Hence, the recipient needs to check that

$$ channel.amount \le value(ticket) + \sum_{t \ \in \ stored} value(t)$$

In addition, as tickets are issued with an incremental serial number, the recipient must check that $ticket_i.index > \max(ticket_{i-1}.index,0)$ and refuse the ticket otherwise.

It remains to be shown that the ticket issuer indeed knows any $response$ which solve $ticket.challenge$. This is especially relevant if the ticket issuer was given the challenge by a third party, i.e., the creator of a mixnet packet. This will be covered in Section \ref{sec:incentives:proofofrelay}.

\paragraph{Validation of Unlocked Tickets}
\label{sec:tickets:validation:unlocked}

Once the $response$ to $ticket.challenge$ is known, i.e., after receiving a packet acknowledgement, the node can determine whether the ticket is going to be a winner. To check this, the node first computes the next $opening$ to the current value $commitment$\footnote{TODO: Properly define this value} stored in the smart contract and checks whether

$$ \mathsf{keccak256} ( \ \mathsf{keccak256}(ticketData) \ || \ solution \ || \ opening \ ) < ticket.winProb $$

If true, the node can consider the ticket to be a winner and store it for later use. If the ticket turns out to be a loser, there is no added value to it and the node can safely drop it. Note that losing tickets are an integral part of the mechanism and do not reduce the average payout to the ticket recipient. This is because $value(ticket)$ is given by the expected value and hence the asymptotic payout does not change.