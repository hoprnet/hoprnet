\subsection{Assessment}
\label{sec:intro:assessment}

Although there are numerous projects attempting to solve the problem of online privacy, most fail to satisfy the design principles and security goals outlined in Section \ref{sec:intro:securitygoals}.

\paragraph{Privacy} The metadata privacy properties that a mixnet like HOPR provides are significantly beyond what VPNs, dVPNs, or even Tor can deliver. VPNs and dVPNs have single points of trust and failure. Tor and I2P have been shown to be subject to a large variety of de-anonymizing attacks, which a mixnet like HOPR is resilient against. The HOPR mixnet unlinks sender and recipient even when faced by powerful global passive adversaries (GPAs) which can monitor every packet in the network. This is achieved by packet transformation, mixing, and introducing delays before forwarding a packet, as well as bandwidth overhead in form of cover traffic and packet padding.

\paragraph{Decentralization} Most projects presented above have decentralization as a core goal (with the exception of VPNs, where the service is provided by a centralized entity, at the expense of user privacy). However, many projects lack the impetus to scale, which prevents them from leveraging many of the advantages which come with a decentralized network, such as privacy through obscurity, while retaining all of the challenges, such as coordination and consensus issues. It is our opinion that the only way to solve the scalability problem is with the introduction of a robust incentivization scheme for node runners.

\paragraph{Incentivization} Many projects referenced in this section lack incentivization of any kind. VPNs incentivize the service provider, but since this is a centralized entity this fails the privacy condition.

Bandwidth providers in dVPNs share their resources and are granted tokens accordingly as payment for their services. For example, Mysterium \cite{mysterium}, an open-source dVPN built on top of a P2P architecture, uses a smart contract on top of Ethereum to ensure that the VPN service is adequately funded. However, the resultant liability risk for dVPNs means these incentives are likely to be insufficient.

Tor and I2P rely on donations and government funding, which only covers the cost of running a node, not any additional reward. This has discouraged volunteers from joining these networks and the number of relayers in both networks has stayed mostly static in recent years, even as usage and the demand for privacy has risen sharply.

Mixnet designs are generally unconcerned with the problem of incentivization, and most existing implementations rely on a group of volunteer agents who lack incentives to participate. However, it is possible to add incentivization on top of a mixnet design, which is what HOPR does. Some authors have proposed adding digital coins to packet delivery in mixnets \cite{MixnetCoins}. However, here the anonymity provided by the mixnet acts as a double-edged sword: since there is no verification for whether a packet arrives at its final destination, and node identities are kept secret by the mixnet, selfish nodes can extract payment without performing their relaying task or without passing on acknowledgements to the next node. Thus a naive implementation of this approach does not actually provide an incentive at all.

It is in this last area that HOPR provides its main innovation: HOPR's proof-of-relay mechanism ensures that node runners are only paid if they complete their relaying duties. The challenge is to enforce this without publicizing data which would allow an adversary to break the anonymity of the network, and to provide both incentivization and anonymity without introducing unacceptable latency and bandwidth costs. The remainder of this paper will explain how this is achieved.\footnote{TODO: Clarify the preceding from two fronts: i) More incisive data and/or argumentation about how HOPR overcomes these shortcomings from competing approaches and ii) Demonstrate why it's important for HOPR to exist as an independent entity, rather than an augmentation to one of these other approaches.}
