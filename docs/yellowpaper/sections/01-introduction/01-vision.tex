\subsection{HOPR Design Ethos}
\label{sec:intro:vision}

HOPR is built with the following principles in mind: that privacy must be an integral part of the design process (privacy by design); that privacy cannot be reliably achieved in centralized systems;\footnote{TODO: Add reference} and that decentralized systems must be properly incentivized to provide privacy at a sufficient scale and level of reliability. The following section explains these three principles in more detail, as well as how HOPR meets them.

\subsubsection{Privacy by design}
Privacy by design is an approach to systems engineering which considers privacy throughout the entire engineering process, not merely as an afterthought.\footnote{TODO: Add reference} The internet is a public good – a digital commons that should be safe and secure to use for all users. However, it is impossible to provide such privacy using current internet infrastructure, \footnote{TODO: Add reference, or augment https://derp.hoprnet.org to provide a more robust demonstration of this claim} which attaches little or no importance to metadata privacy, and indeed in many cases relies on copious amounts of metadata to be made public in order to function. Therefore, new infrastructure with a focus on privacy must be created on top of the existing internet. HOPR provides an essential part of that infrastructure, and has been built with privacy as its foremost goal.

\subsubsection{Decentralization}
The internet is not private by design, although it is decentralized by design. In the early days of the internet, this decentralization provided a certain measure of privacy, by placing control in the hands of individual users. However, internet users must increasingly interact with services provided by central authorities. These central authorities control the privacy of individual users, often without their full informed consent.

In particular, interactions with third-party service providers regularly entail a loss of transport-layer privacy. Links can often be drawn between the user and the service provider, either by the service provider themselves, or by a third party who can observe public metadata revealed during the interaction or pressure the service provider to divulge stored data. With a sufficient number of such links, a profile can be constructed of a user's online activity.\footnote{TODO: Add reference}

This runs contrary to the requirements of privacy as a fundamental human right, since often the only truly private option is to abstain from using such services entirely. To provide privacy as a fundamental feature on top of the internet, a decentralized infrastructure is required.

The HOPR protocol runs on nodes within a decentralized network, therefore ensuring that the network is independent. No single entity can influence its development or manipulate its performance to their advantage. It also makes the network resilient, able to keep running even if a majority of nodes are damaged or compromised and very difficult, if not impossible, to shut down.

\subsubsection{Incentivization}
Although users are incentivized to use privacy technologies by the privacy they provide, most privacy technologies have no baked-in incentive framework for infrastructure providers: rewards either flow to a centralized service provider (and thus the service is not truly private, since the service provider will gather transport-level information on user activity) or the providers in a decentralized network receive no direct rewards. This constrains the growth of the network and limits its scope. In systems which leverage privacy through obscurity, this reduced scale compromises the privacy of the entire network, since fewer users means less data to confound observers. Although blockchains provide a way, for the first time, to incentivize a decentralized network without introducing a centralized entity responsible for payment provision, it is challenging to leverage blockchain technology without compromising either the privacy of network members or the reliability of the network. In brief, if node runners can rely on the anonymity of the technology they are providing, they can potentially use it to claim rewards without properly fulfilling their duties as a node runner. On the other hand, if node runners are required to interact with a blockchain to prove their service provision and claim their reward, the resultant on-chain data is liable to provide a permanent and growing source of metadata which can be used to erode the users' privacy.

HOPR's proof-of-relay mechanism threads this needle, and is the major innovation of the HOPR protocol. Every HOPR node can receive payment for each packet they process and forward, but only once their data payload arrives at the next node. Payments are probabilistic and cheat-proof, which ensures nodes do not (de-)prioritize other nodes and/or packets. To maximize received incentives, node operators must ensure good network connectivity, node availability, computational capacity, and committed funds (to reward other nodes). These combined incentives promote a broad, robust, and reliable network. At the same time, the probabilistic nature of payments obscures links between on-chain data and node runners.

